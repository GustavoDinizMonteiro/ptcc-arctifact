\documentclass{article}
\usepackage[utf8]{inputenc}

\title{Network of Favors in P2P cloud federation}
\author{Gustavo Diniz Monteiro \\ \href{gustavo.monteiro@ccc.ufcg.edu.br, dinizmonterogustavo@gmail.com} }
\date{November 2018}

\begin{document}

\maketitle

\section{Introduction}

Master: Francisco Vilar Brasileiro \\ \href{fubica@computacao.ufcg.edu.br}

\section{Abstract}
The objective of this work is to present a form of justice implementation and cooperation incentive in P2P systems based only on direct reciprocity model called Network of Favors. This model is thought for small and medium size networks and for scenarios where
repeated interaction among peers is more likely and problems with no cooperative peers that commitment to justice and availability for participatory members who need resources and causes members to leave the network, also this solution is very lightweight and it does not involve any issues of trust, resting on the first hand knowledge acquired by each peer through direct interaction with other peers.

\section{The Problem}

\section{Objectives}

\section{Chronogram}
\subsection{Activities}
This work have 3 stages, the implementation of the Network of Favors as a component of the Fogbow project, this step can be separated in two steps, the implementation of business login and after of the integration tests with the Fogbow ecosystem, after that creation of a experimentation model and metrics for test in by simulation and too in production application and get the results, the last activite is with the resultss of experiment make a study, writing a writing a paper with the study results.

\end{document}
