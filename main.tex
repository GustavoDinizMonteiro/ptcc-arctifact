\documentclass{article}
\usepackage[utf8]{inputenc}

\title{Network of Favors in P2P cloud federation}
\author{Gustavo Diniz Monteiro \\ \href{gustavo.monteiro@ccc.ufcg.edu.br, gustavo.d.monteiro@icloud.com, dinizmonterogustavo@gmail.com} }
\date{November 2018}

\begin{document}

\maketitle

\section{Introduction}

Master: Francisco Vilar Brasileiro \\ \href{fubica@computacao.ufcg.edu.br}

\section{Abstract}
The objective of this work is to present a way of justice implementation and incentive to share resources in P2P systems based only on the direct reciprocity model called the Favors Network. This model is designed for small and medium-sized networks and for scenarios where repeated interaction between peers is more likely and problems with non-cooperative members who compromise justice and resource willingness for participatory members who need and make members leave the network, in addition to everything the solution is very light, is designed to be an element pluggable and does not involve any trust issues, using first hand knowledge gained by each member through direct interaction with other peers.

\section{The Problem}

\section{Objectives}

\section{Schedule}

The development of this study is planned to take place between January and June of the referent year of 2019 and will be divided into 4 main stages:
\subsection{Development of application business logic}
	In this step, components will be developed that address the business logic of the NOF service, including requisition middleware, federation member authentication services and favor accounting service, as well as favour allocation and resizer flows. quotas. be responsible for the implementation of justice, at the same time, the unit tests related to its components will be developed. The planned deadline for this activity is to do it throughout the month of January and February.
\subsection{Development of integration tests}
    In this stage, the integration tests of the tool of the NOF component with the tool of provisioning and administration of federations of Fogbow hybrid clouds will be developed. The integration tests will be developed to test all of the most common streams of resource requests, Fogbow provision, special non-participant member detection cases, and resiliency in cases of error. The deadline for this activity is to do so throughout the month of March.
\subsection{Write the paper}
    This last phase will be dedicated to the writing of the article, where all the decisions of software engineering and re results obtained with the NOF tool will be discussed for this work, which includes all the revisions of the article and possible revisions of architecture and / or implementation in the NOF it was planned to take 3 months, from April to June.

\subsection{Activities}
This work have 3 stages, the implementation of the Network of Favors as a component of the Fogbow project, this step can be separated in two steps, the implementation of business logic and after of the integration tests with the Fogbow ecosystem, after that creation of a experimentation model and metrics for test in by simulation and too in production application and get the results, the last activity is with the results of experiment make a study, writing a writing a paper with the study results.

\end{document}
